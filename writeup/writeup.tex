%
%   latex writeup for Peerchat
%   6.858 Final Project
%   December 2013
%   forrestp, ameeshg, kseibert, cbieden
%
%

\documentclass{article}

\usepackage{geometry}
\geometry{letterpaper}

\usepackage{doc}

%\usepackage{graphicx}
%\usepackage{float}
%\usepackage{caption}
%\usepackage{subcaption}

%\usepackage{epstopdf}

%\usepackage{enumitem}
%\setdescription{leftmargin=\parindent,labelindent=1cm}

\title{Peerchat}
\author{
  Forrest Pieper\\
  Will Drevo\\
  Colin Taylor
}
\date{May 6th, 2014}

\begin{document}

\maketitle

\section{Introduction}
\label{introduction}

What we today know as the "internet" was started as a US DARPA military project, a network of nodes distributed geographically across the US in order to ensure fault tolerence in the case of a nuclear attack \cite{?}. Today, ironically, many feel the internet has become too centralized. \\

Something something here \\

\textit{Peerchat} is a distributed, P2P chat system based on the Kademlia DHT \cite{Maymounkov02} system. 

\section{Background and Related Work}

A very similar effort is BitTorrent Chat \cite{?}. 

\section{Implementation}

\subsection{System}
Peerchat is divided into three main subsystems. A distributed hash table based on Kademlia \cite{?} maps a username to an ip address. A 

Separate threads:
\begin{enumerate}
	\item Message Queuing
	\item Message Sender
	\item Connection Acceptor
	\item Periodic Backup 
\end{enumerate}

First, messages are queued by a sender. The user in the chat interface sends a message, which is put into the pending messages queue. 

(message sender desc)

(Connection Acceptor desc)

(Periodic Backup  description here)

\subsection{Protocol}

\subsection{Persistence}

We persist nodes' state by periodically serializing a user's routing table and messaging log to disk. 

\subsection{Offline Usage}

\section{Demonstration}

\subsection{User Registration}
\subsection{User Login}
\subsection{Correctness Testing}
\subsection{Performance Testing}

\section{Future Work}

Security?

\section{Conclusion}

Peerchat is the best, blah, blah.

\begin{thebibliography}{99}
  \bibitem{Maymounkov02}
    %Kademlia paper
   Maymounkov, Petar and David Mazieres
   ''Kademlia: A Peer-to-peer Information System Based on the XOR Metric''
   \textit{Peer-to-Peer Systems. Springer Berlin Heidelberg}, 2002. 53-65
 
\end{thebibliography}

\end {document}
